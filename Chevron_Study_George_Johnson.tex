\documentclass[12pt,letterpaper]{article}

\usepackage[margin=1.5in]{geometry}
\usepackage[english]{babel}
\usepackage[utf8x]{inputenc}
\usepackage{amsmath}
\usepackage{amssymb} 
\usepackage[retainorgcmds]{IEEEtrantools}
\usepackage{graphicx}
\usepackage{tabularx}
\usepackage{subfig}
\usepackage{kpfonts}    % for nice fonts
\usepackage{microtype} 
\usepackage{booktabs}   % for nice tables
\usepackage{bm}         % for bold math
\usepackage{listings}   % for inserting code
\usepackage{verbatim}   % useful for program listings
\usepackage{color}  
\usepackage[colorlinks=true]{hyperref}
\usepackage[colorinlistoftodos, textcolor = black]{todonotes}
\usepackage{natbib}
\usepackage{setspace}
\usepackage{indentfirst}
\usepackage{url}
\usepackage{hyperref}

%Formatting
    \doublespacing
    \setlength {\marginparwidth }{2cm}

%Article
    \begin{document}

%Title
    \title{\textbf{Chevron: An International Success or Failure?}}
    \author{George Johnson\footnote{Economics B.S at University of Houston. Email: GtJohnso@cougarnet.uh.edu}}
\date{January 4, 2025}
\maketitle
\newpage

\section{Argument}
    Chevron, a company often referred to as one of the largest energy suppliers around the world, can be considered a truly global company. Chevron Corp. operates in more than 180 different countries \footnote{\cite{Statista}}, placing it at the forefront of the market. The global context allows chevron not only to control pricing of important market components but also to shape policy and law through its large influence on energy. This level of influence is what gives Chevron the status of a successful globalized energy business. The energy policies of companies on the entry of the market, the adoption of technology, and environmental sustainability display its versatility in such a competitive environment. Yes, Chevron has faced problems, environmental activism, and compliance that still plague the company to this day, but the ability to combat these issues and remain a top player signifies the strength of the company. However, the company was not always a global power as they date back to the beginning of the twentieth century.

\section{Introduction}
    As noted by \textbf{Curley}\footnote{\cite{Curley}}, Chevron commenced operations in 1906 after the joining of two separate oil companies, the Pacific oil Company and the standard oil Company of Iowa. This foundational merger marked the beginning of aggressive business policies undertaken by Chevron’s management in their quest for company growth and expansion through acquisition. In the 1930's, Chevron took advantage of the oil boom in the Middle East, teaming up with Saudi Arabia to exploit the new oil discovery. After the second World War Chevron becomes faster in expansion and the number of chevron acquisitions grew through the years until the twentieth century. One of the significant events came in 1984 when Chevron merged with Gulf Oil, which was one of the largest corporate mergers during that period. This momentum continued on with the acquisitions of Texaco and Unocal in the later decades; rendering excellence to the second place purely behind Exxon-Mobil. These series of mergers and acquisitions enabled Chevron to expand its outreach into many parts of the world and diversified its business to cover the entire petroleum industry, which include upstream, midstream and downstream activities of refining, exploring and marketing of oil. However, this international growth and success comes at a price and these are quite substantial environmental and humanitarian issues.
    
\section{Problems and Implications}
    In the opinion of \textbf{McRobert and Shay}, Chevron is one of the companies that has faced many suits in regard to environmental degradation in countries where its business is. As an outcome, such procedures as Corporate Social Responsibility (CSR) appeared in order to balance social and environmental consequences of the business, which was mainly the result of pressure from green and other local communities affected. However, despite these moves or attempts, in many instances, Chevron has abused legal loopholes to evade responsibility. For example, Chevron has contended in a Canadian court that Chevron Canada is a distinct corporation that is not responsible for any environmental harm arising from its activities in Ecuador\footnote{\citeauthor{David}}. This case serves to demonstrate the power of Chevron not just as a business corporation, but as a legal construct that defines the understanding of business accountability and responsibility in the world.

    The Ecuadorian case highlights in broad terms the difficulties in applying the concept of CSR\footnote{\citeauthor{Cass}}: how Chevron, for example, is able to escape such accountability in other countries as Ecuador. Implementing them presents a structural tension between the intuitively accepted social responsibilities and the structures of incentives and defenses that enable a corporation to escape the consequences of its actions. Transnational cases such as Chevron’s rise again the issues as to whether in fact law is able to do justice to the world’s multinational corporations or they are able to perpetuate abuses like the corporate veil to hide behind. The delay in resolution of legacy contamination catastrophe such as Bhopal gas tragedy and more so Exxon Valdez oil spill strengthens the case for fundamental change in systems. It is indeed very much the case that all these emphasize the necessity of consolidation of greater partnership among states, civil society and the business world to achieve greater accountability and transparency. They point out the need to bring in a more watertight regulatory framework that narrows legal loopholes and raises the level of responsible participation of company members in remediation. What this litigation case of Chevron demonstrates is the gap in the implementation of corporate social responsibility and gives a challenge to all CSR stakeholders to assess the effectiveness of the CSR s.

\section{Analysis}
    The evolution of a company, such as Chevron’s evolution to a global dominator of energy resources, can be analyzed in terms of corporate identity and corporate strategy. A significant part of the company’s corporate identity has been its strength to establish itself as a global dominant and a diversified energy supply , operating in more than 180 countries and economically, politically, and legally engaged. This particular approach captures its intention to be able to be actively involved in the oil and gas industry in various regions of the world in which the company operates. Yet, Chevron’s corporate identity is not only undermined by environmental issues, which begs the question of the company’s adherence to the guidelines of Corporate Social Responsibility. The disjunction between its selfpresentation as a good corporate citizen practice and its legal battles so that it is not held liable, as in the case of Ecuador, depicts an important shortcoming in the company’s strategy of sustainability.

    Chevron started its rise and expansion by acquiring Gulf Oil and, subsequently, Texaco and Unocal. This assertive expansion policy became explosive and internationally globalized the company. By means of this strategy the company was able to retain its market position, protect its range of operations, and compete successfully in a rapidly changing industry. But one and the same focus on growth and profits as the strategy resulted in issues if the company’s social and environmental impacts were looked into. Such heavy reliance on legal constructs such as the corporate veil or limited liability for the abuse of joint stock protects necessitates a shift of strategy where CSR is central to all business processes rather than peripheral. This change would not only address increasing worldwide criticism but also consolidate the over-all firm status and its endurance.



\section{Conclusion}
    Chevron has competitive strengths such as a robust corporate philosophy which gives a strong consistent international profile, however, the future soundness of the company will depend on how they cope with escalating demands for accountability and sustainability.  Chevron’s ability to reconfigure self-image within the limits of multinational indicator of CSR and devise a strategy which does not compromise the emphasis on the profitability with CSR, will be important. Otherwise, it may suffer from loss of image, regulatory problems, or reduced trust from its stakeholders, not the least in a time of increasing concern for the environment. Chevron has the competitive advantage of clarity if working with all relevant actors, and the ability to operate within a sustainability framework; assuming these are the predetermined objectives, nothing should stop Chevron from assuming a position of leadership within the global scope.


\newpage
\bibliographystyle{apalike}
\bibliography{Citations}
\cite{Curley}
\cite{Statista}
\cite{David}
\cite{Cass}
\end{document}